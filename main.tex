\chapter{Eigensolver for General One-body Potentials}
\label{c:inhalt}

%%%%%%%%%%%%%%%%%%%%%%%%%%%%%%%%%%%%%%%%%%%%%%%%%%%%%%%%%%%%%%%%%%%%%%%%%%%%%%%%


Bei ihrer Abschlussarbeit handelt es sich um eine
  wissenschaftliche Arbeit, die auch entsprechenden
  Qualit"atsanspr"uchen gen"ugen muss.
  \begin{itemize}
  \item Verwenden Sie keine umgangssprachlichen Formulierungen.
  \item Achten Sie darauf, alle Aussagen, die Sie machen, durch
    entsprechende Argumente oder Literaturverweise zu untermauern.
  \item F"uhren Sie vor der Abgabe eine Rechtschreibpr"ufung durch.
    Ein g"angiges Werkzeug hierf"ur ist beispielsweise
    \texttt{aspell}, dessen Verwendung auch in Editoren wie Emacs
    vorgesehen ist.
  \end{itemize}


Kleiner Test ob das Komplieren funktioniert
\section{Methods for Eigensolver}
\section{Approaches to Generate Initial Guess Functions}

\section{Basis Functions for Harmonic Oscillator}
\section{Basis Functions for General Potentials}
\subsection{General Basis Functions}
\subsection{Examples}